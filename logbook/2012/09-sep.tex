\chapter{September}
\section{05-06 September 2012}
Notes taken from \url{obitko.com/tutorials/genetic-algorithms/index.php}
\subsection{Basic Outline}
\begin{enumerate}
\item Generate a random population (suitable to your problem)
\item Evaluate the fitness of the chromosomes in the population
\item Select two parents according to their fitness
\item Crossover(mate) them to generate offspring
\item Potentially mutate the child. 
\item Put the children into the population
\item Evaluate the fitness of the children
\item See if we have improved things
\item Rinse, repeat
\end{enumerate}

\subsection{Types of Selection}
\subsubection{Roulette Wheel}
Parents are selected according to the fitness. Those that are more fit have a 
better chance to get selected. This method can have problem when the fitness 
varies greatly within the population.

\subsubsection{Rank Selection}
First rank the population, and then every chromosome receives a fitness from 
the rank. The worst will have fitness 1 and the best a fitness of N, where
N is equal to the number of chromosomes in the population.

\subsubsection{Steady State Selection}
Pick some with a good fitness to reproduce, and replace the ones with the lowest 
fitness by these offspring. The remaining chromosomes survive to the next iteration.

\subsubsection{Elitism}
First copy over some of the best to the new population. Then proceed to use one 
of the aforementioned methods.

\subsection{Binary Encoding}
\label{subsec:binary}
Gives many possible chromosomes to choose from. However, this is sometimes not 
a natural way to encode the problem. See, Example \ref{subsubsec:knapsack}. 
\subsubsection{Crossover}
\begin{enumerate}
\item Simple point $\rightarrow$ Copy part from A, rest of B.
\item Two Point  $\rightarrow$ Copy part from A, part from B, rest from A.
\item Uniform  $\rightarrow$ Randomly copy bits
\item Arithmetic  $\rightarrow$ Perform some arithmetic operation.
\end{enumerate}
\subsubsection{Mutation}
Random select bits to be flipped

\subsection{Permutation Encoding}
This is good for ordering problems. A string of numbers/letters represents the 
data. One must take care during crossover and mutation to ensure you are left 
with a real possible solution. See, Example \ref{subsubsec:salesman}.
\subsubsection{Crossover - Single Point}
Copy to the point from A, scan B and add those that do not already appear. 
\subsubsection{Mutation}
Randomly choose two numbers and exchange.

\subsection{Value Encoding Encoding}
Every chromosome is a string of values, useful where binary encoding would be 
difficult. Good for some special problems but methods of crossover and mutation 
are highly specific to the problem. See, Example \ref{subsubsec:neural}.
\subsubsection{Crossover}
All crossovers from Binary Encoding(Sec. \ref{subsec:binary} are useable.
\subsubsection{Mutation}
For real numbers add or subtract a small number. 

\subsection{Tree Encoding Encoding}
Every chromosome is a tree of objects such as functions or commands in a 
programming language. This is good for evolving programs. LISP uses it quite 
a bit. See, Example \ref{subsubsec:func}.
\subsubsection{Crossover}
One point is selected and parents are joined at that point. 
\subsubsection{Mutation}
Change the operator, number, selected nodes, etc. 
\subsection{Example Problems}
\subsubsection{Knapsack Problem}
\label{subsubsec:knapsack}
Things have been given a size, backpack has set size, put all things possible 
into backpack. Each bit says if the thing is in the pack. This example is best 
encoded with a binary representation.

\subsubsection{Traveling Salesman Problem}
\label{subsubsec:salesman}

\subsubsection{Neural Networks}
\label{subsubsec:neural}

\subsection{Recommended Values}
\begin{description}
\item[Crossover Rate] 80-90\% some show 60\%
\item[Mutation Rate] Low - 0.5 - 1.0\%
\item[Population] 20-30, sometimes larger. This really depends on the encoding 
  and the number of alles. For example, if the chromosomes have 32 bits, then 
  choose a population with about 32 chromosomes. 32 chromosomes would be way too 
  many for a 16-bit chromosome.
\item[Selection] Depends on the problem
\item[Encoding] Depends on the problem
\item[Croccover/Mutation] Depends on the problem.
\end{description}

\section{07 September 2012}
We are definitely goign to want to use the tree encoding for the function finding. 
It makes me wonder if this has been done before. Michael confirms that this is the 
case, but we will be using a new fitness test. 
